\index{dpss@\texttt{dpss}!executable}
Currently, only the \texttt{dpss} module has been implemented as a command-line executable.  The binary prints the sequences and eigenvalues directly to the terminal.  On Unix-based systems, this output can be re-directed to a file in the standard way.  The format of the output is compatible with a \textsc{Matlab}/Octave script file, producing the two arrays: \code{dps_seq} for the sequences, and \code{lambda} for the eigenvalues.  The following is the help file, which can be accessed with \texttt{dpss --help}:
{\small
\begin{verbatim}
  Usage: dpss n nW [[seql] sequ] [interp_method [interp_base]] [trace]

  `dpss' calculates and prints a set of "discrete prolate spheroidal sequences"
  (aka Slepian Sequences) and their corresponding concentrations of energy in
  the normalized bandwidth [-W, W].  These sequences are typically used in 
  multitaper spectral analysis.

  Examples:
  dpss 12 3 2        returns the first two DPSSs of length 12 and time-bandwidth
                     product nW=3, along with their concentrations.
  dpss 12 3 3 5      returns the third through fifth DPSSs and concentrations.
  dpss 8388608 2.5 spline 32768   calculates the first five DPSSs of length 2^15,
                                  nW=2.5, then interpolates to length 2^23 using 
                                  natural cubic splines, re-normalizes and
                                  estimates the new energy concentrations.

  Required Parameters:
    n     Length of the dpss sequences to be computed
    nW    Time-bandwidth product for the sequences.  By definition, the first
          dpss maximizes the concentration of energy in the normalized frequency 
          range f in [-W,W]. Typical values are nW = 2, 2.5, 3, 3.5 and 4.

  Optional Parameters:
    sequ             The index of the upper dpss to be calculated.  Indices must
                     fall in the range [1, N].  If `seql' is not defined, then
                     this is the total number of sequences.  By default,
                     sequ = floor(2nW). 
    seql             The index of the lower dpss to be calculated.  This must
                     fall in the range [1, sequ].
    interp_method    Interpolation method to employ when creating the sequences.
                     By default, dpss will calculate sequences up to length
                     n = 2^17-1 without interpolation.  For larger sequences, 
                     intermediate values must be interpolated.  Accepted methods
                     are `linear' and `spline' (without quotes).  By default,
                     interp_method=spline, which uses natural cubic splines.
    interp_base      Length of the sequences from which to interpolate.  An 
                     interpolation method must be supplied to use this option.
                     By default, interp_base = min( n, 2^17-1 ).
    trace            trace = `trace' (without quotes) prints the method and 
                     computation parameters to the command window.

  See the `mtpsd' Documentation for further details.
\end{verbatim}
}