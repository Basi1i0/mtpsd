The intent of this documentation is two-fold:
\begin{itemize}
    \item to describe the use of the classes and methods contained in the 
        \texttt{mtpsd} library, and
    \item to explain the underlying mechanisms and the reasoning behind them.
\end{itemize}
In my (short and limited) experience, I've learned to be wary of academic tools
that don't explain precisely the way they work behind-the-scenes.  When 
something isn't behaving as expected, one is often stuck examining the source 
code, trying to decipher the algorithms the author used (or \emph{tried to} 
use).  It is my hope that this document is clear about how each method does 
what it does.  Unfortunately, this meant I had to include a Theory section, 
where I could put the equations involved.  For this section, it is assumed 
that the reader is somewhat familiar with frequency analysis, and has at least 
heard of David Thomson's multitaper technique.  If not, excellent sources of 
information are Thomson's seminal paper on the multitaper method 
\cite{thomson:multitaper}, and Percival and Walden's fundamental text on 
spectral analysis \cite{percival:multitaper}.

The original motivation for this project was, unfortunately, the result of an 
academic tool not behaving as expected.  While taking a course on statistical 
signal processing (taught by Thomson), I found that the confidence intervals 
produced by \textsc{Matlab} for an adaptively weighted spectrum estimate seemed
to match too closely to those for uniform weighting.  At the time, I just found
it curious, but decided not to investigate.  It wasn't until recently, when I 
needed the functionality in the open-source GNU Octave, that I had cause to 
look into it.

The Octave signal processing toolbox does not include the multitaper technique,
and I was not able to find a suitable (open-source) replacement.  So, I began 
coding my own, comparing results with \textsc{Matlab} to make sure the two were
consistent.  For eigenvalue weighting, the two results were close, but would 
not match exactly.  For adaptive weighting, my confidence intervals were quite 
a bit larger, which reminded me of Thomson's course.  This led me to examine 
the source code for \textsc{Matlab}'s \texttt{pmtm} function, to see exactly 
what was being returned.  I found the following `bugs':
\begin{itemize}
    \item For eigenvalue weighting with $K$ tapers, the weights are 
        approximated as $w_k$ = $\lambda_k/K$, when they should be 
        $w_k$ = $\lambda_k/\sum_i\lambda_i$.
    \item Confidence intervals are computed as if equal weighting was used, 
        setting $\nu=2K$, where $\nu$ is the degrees of freedom in the estimate.
\end{itemize}
The first `bug' isn't much of a problem as long as the number of tapers is less
than twice the time-bandwidth product.  However, being a student of
mathematics, it bothers me that the sum of weights used in a weighted average 
is not equal to one.  The second bug, however, \emph{is} a problem.  The 
purpose of adaptive weighting is to adjust weights in order to reduce 
broad-band bias.  This makes the weights frequency-dependent, and reduces the
degrees of freedom in some areas of the estimate.  In heavily biased regions, 
$\nu$ can be near 2, which can differ significantly from $2K$.  There is a 
trade-off: the bias is reduced, but the confidence interval is widened.  If one
simply trusted the results of \texttt{pmtm}, there would be no apparent cost to
reducing bias.  In fact, since the adaptively weighted estimate is lower than 
the equally weighted one in biased regions, the interval reported by 
\texttt{pmtm} is actually narrowed.  This \emph{contradicts common-sense}: you 
should not have more confidence in a result that uses less information.  These
errors were the inspiration for the development of a publicly-available 
library.  I chose to use C++ so I could use the methods in another project.  
And so began \texttt{mtpsd}: the Multi-Taper Power Spectral Density estimator.

I should mention another publicly-available package, written in Fortran 90 by 
Germ\'an A.~Prieto: \href{http://wwwprof.uniandes.edu.co/~gprieto/software/%
mwlib.html}{\texttt{mwlib}}.  I found the library to work well, for the most
part.  Again, for eigenvalue weighting, the weights are approximated as 
$w_k$ = $\lambda_k/K$, which is only appropriate when $K$ is less than twice 
the time-bandwidth product.  Also, the degrees of freedom are computed in a 
non-smooth fashion:
\begin{align*}
    \alpha_k & = \dfrac{w_k\sqrt{K}}{\sqrt{\sum_{i=0}^{K-1} w_i^2}} %
        & \nu & = 2\sum_{k=0}^{K-1} \min\{\alpha_k,1\}.
\end{align*}
This is inconsistent with the assumption that the spectrum estimate follows a
scaled $\chi^2_\nu$ distribution, as put forth in \cite{percival:multitaper,%
thomson:multitaper,thomson:lecture}. Specifically, the variances do not agree.
Also, in the \texttt{mwlib} package, the confidence interval is computed using
a jack-knife, leaving out one eigenspectrum at a time.  This may not be 
suitable for a small number of tapers.  The tool does, however, have a useful 
methods for removing line frequencies and reshaping the spectrum estimate, 
which more than make up for the flaws mentioned.

The \code{mtpsd} library is available for public use, released under GPLv3.  
Feel free to modify the code for your own purposes.  It has been tested on 
64-bit Windows 7 with MSYS/MinGW64, and on 64-bit Ubuntu 12.10.  The Octave
extensions have only been tested on 32-bit versions of Octave.  If you have 
problems compiling, or have found a bug, please let me know so I can fix the 
issue.
\smallskip

\hfill $\sim$ Antonio\hspace*{4ex}
\bigskip

\vfill
\noindent \parbox{\textwidth}{
C.~Antonio S\'anchez\\
Master of Mathematics (2010), University of Waterloo\\
Bachelor of Science (2007), Queen's University\\
\href{mailto:antonio@eigenspectrum.com}{antonio@eigenspectrum.com}
}
\smallskip
