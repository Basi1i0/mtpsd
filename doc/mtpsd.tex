\index{mtpsd@\texttt{mtpsd}|(}There are two main ways to use the \texttt{mtpsd} library.  The first is through the use of an object.  The \code{mtpsd.h} header defines the \code{mtpsd} class, which can be used to compute any of the multitaper spectrum estimates and properties described in Section \ref{sec:theory}.  All of the internal variables are protected, so must be accessed through provided accessors.  One special accessor is defined, \code{mtpsd::pS()}, which returns a \code{const double} pointer to the protected spectrum array, allowing more direct access.  The class and its use are described in Section \ref{sec:basic}.

The second way to use the library allows for more control over memory and access to intermediate variables.  A set of functions are defined that accept and fill user-defined arrays.  For multi-dimensional arrays, row-major format is assumed.  This advanced use is covered in Section \ref{sec:advanced}.

\subsection{Basic Use: the \texttt{mtpsd} Class \label{sec:basic}}

Listing \ref{lst:basic} shows the most basic construction and use of an \code{mtpsd} object.  \index{mtpsd class@\texttt  {mtpsd} class|(}
\begin{lstlisting}[label=lst:basic,caption=Basic example of the \texttt{mtpsd} class]
#include "mtpsd.h"              /*@\lstomitnum@*/
...                             /*@\lstskipnum{8}@*/
{
    double *x;          // time series
    uint_t n;           // length(x), (unsigned int)
    double nW;          // time-bandwidth product   /*@\lstomitnum@*/
    ...                         /*@\lstomitnum\lstskipnum{7}@*/
    
    mtpsd<double> spectrum(x, n, nW);
    try{
        spectrum.compute();
    }
    catch(ERR e){
        printf("ERROR: %s\n", e.getmsg());   
    }                           /*@\lstomitnum@*/
    ...                         /*@\lstomitnum\lstskipnum{7}@*/
    
    printf("S=%f, f=%fHz \n", spectrum(i), spectrum.freq(i));    /*@\lstomitnum@*/
    ...                         /*@\lstskipnum{8}@*/
}
\end{lstlisting}\index{mtpsd class@\texttt  {mtpsd} class!compute@\texttt{compute()}}
\smallskip

\noindent In the example, the \code{spectrum} object contains all information pertaining to the multitaper estimate.  There are several things to note:
\begin{itemize}
    \item \code{mtpsd} is an \emph{template class}.  The library is configured to compile for types \code{double} and \code{fftw\_complex} (\code{double \_\_complex\_\_}).  This must correspond to the data type of the time-series \code{x}.
    \item The \code{compute()} method may throw an error of type \code{ERR}\index{ERR class@\texttt{ERR} class}.  This class is defined in \code{simple\_error.h} (see Section \ref{sec:simpleerr}).  The error class only has one property: an error message, accessible through \code{ERR::getmsg()}.
    \item All lengths and indices are of type \code{unsigned int}.  A short-form for this type, \code{uint\_t}\index{uint\_t@\texttt{uint\_t}}, is defined in the \code{mtpsd.h} header.
    \item This basic constructor assumes adaptive weighting, a sampling frequency of 1 Hz, and uses $K=\max\{\lfloor2nW\rfloor-1,2\}$ data tapers.
\end{itemize}
The desired spectrum information can be extracted through one of the object's accessors, which are described in Table \ref{tbl:mtpsdaccess}.  The public definition of the \code{mtpsd} class is given in Listing~\ref{lst:mtpsdclass}.
\smallskip

\index{mtpsd class@\texttt  {mtpsd} class!definition|(}
\begin{lstlisting}[label=lst:mtpsdclass,caption=The \texttt{mtpsd} class]
template <class T>
class mtpsd {
    // constructors/destructor
    mtpsd(T *data, uint_t n, double nW);                       /*@\label{lst:mtpsdconstruct1}@*/
    mtpsd(T *data, mtpsd_workspace work);                      /*@\label{lst:mtpsdconstruct2}@*/
    mtpsd(T *data, const double *tapers, const double *lambda, /*@\label{lst:mtpsdcustomtaper1}\lstomitnum\lstskipnum{99}@*/
             uint_t n, uint_t K);                              /*@\lstskipnum{-100}@*/
    mtpsd(T *data, const double *tapers, const double *lambda, /*@\label{lst:mtpsdcustomtaper2}\lstomitnum\lstskipnum{99}@*/
             mtpsd_workspace work);                            /*@\lstskipnum{-100}@*/
    ~mtpsd();
    
    // computation routine
    void compute();

    // accessors
    double operator()(uint_t i);           
    const double* pS();
    double operator()(uint_t k, uint_t i);
    fftw_complex eig_coeff(uint_t k, uint_t i); 
    double wt(uint_t k, uint_t i);       
    double freq(uint_t i);                
    double dof(uint_t i);                          
    double conf_int(uint_t i, CONF_BOUND side, double p); 
    double conf_factor(uint_t i, CONF_BOUND side, double p);
    double lambda(uint_t k);
    double taper(uint_t k, uint_t i);
    double F_stat(uint_t i);  
    double F_thresh(uint_t i, double p); 
    bool F_test(uint_t i, double p); 
    uint_t length();  
    uint_t size(int dim);
    mtpsd_workspace getinfo();
};
\end{lstlisting}\index{mtpsd class@\texttt  {mtpsd} class!compute@\texttt{compute()}}
\index{mtpsd class@\texttt  {mtpsd} class!definition|)}

\begin{table}
    \caption{ Description of \texttt{mtpsd}\index{mtpsd class@\texttt{mtpsd} class!accessors} accessors \label{tbl:mtpsdaccess} }
    \vspace*{-0.75em}
    
    \begin{tabular}{c}
        \hline\hline\\[-0.75em]
        \small
        \parbox{\textwidth-12pt}{
            \vspace*{-0.5em}
            \settowidth{\bodyindent}{\texttt{mtpsd\_workspace}}
            \begin{compactcodelist}
                \codeitem{double}{operator()(uint\_t i)}
                    Returns $\hat{S}(f_i)$, the power at the $i$th frequency (starting at $f_0=0$).
                \codeitem{const double*}{pS()}
                    Returns a pointer to the $\hat{S}$ array, so the data can be accessed directly.
                \codeitem{double}{operator()(uint\_t k, uint\_t i)}
                    Returns $\hat{S}_k(f_i)$, the $i$th value of eigenspectrum $k$. 
                \codeitem{fftw\_complex}{eig\_coeff(uint\_t k, uint\_t ii)}
                    Returns $\hat{J}_k(f_i)$, the $i$th value of eigencoefficient $k$.
                \codeitem{double}{wt(uint\_t k, uint\_t i)}
                    Returns the weight factor $w_k(f_i)$, where $\hat{S}=\sum w_k\hat{S}_k$.
                \codeitem{double}{freq(uint\_t i)}
                    Returns $f_i$, the $i$th frequency in Hz.  If the sampling frequency has not been specified, then $f$ is scaled to lie in the range $[0, 1]$.
                \codeitem{double}{dof(uint\_t i)}
                    Computes and returns $\nu(f_i)$ the equivalent degrees of freedom of $\hat{S}(f_i)$, assuming $\hat{S}\sim \frac{1}{\nu}S\chi^2_\nu$.
                \codeitem{double}{conf\_int(uint\_t i, CONF\_BOUND side, double p)}
                    Computes and returns the $p\times100\%$ confidence bound at frequency $f_i$, where \code{side} is either \code{UPPER} or \code{LOWER}.
                \codeitem{double}{conf\_factor(uint\_t i, CONF\_BOUND side, double p)}
                    Computes and returns either the lower or upper confidence factor $\nu/Q_\nu$.  The corresponding confidence bound is obtained by multiplying this factor by $\hat{S}$. Recall that for equal and eigenvalue weights, the confidence factor is independent of frequency.  
                \codeitem{double}{lambda(uint\_t k)}
                    Returns $\lambda_k$, the eigenvalue (energy concentration) of the $k$th taper.
                \codeitem {double}{taper(uint\_t k, uint\_t i)}
                    Returns $h_k(t_i)$, the $i$th value of the $k$th taper.
                \codeitem {double}{F\_stat(uint\_t i)}
                    Computes and returns the F-statistic at frequency $f_i$.
                \codeitem {double}{F\_thresh(uint\_t i, double p)}
                    Computes and returns the $p\times100\%$ threshold for the F-test at frequency $f_i$.  Recall that for equal or eigenvalue weights, this threshold is independent of frequency.
                \codeitem {bool}{F\_test(uint\_t i, double p)}
                    Computes the F-statistic and threshold, and returns \code{true} if the threshold is surpassed (i.e.~the frequency is deemed significant).
                \codeitem {uint\_t}{length()}
                    Returns $N$, the discrete length of the spectrum estimate (length of the frequency grid).
                \codeitem {uint\_t}{size(int dim)}
                    If \code{dim}=0, returns $N$.  If \code{dim}=1, returns $K$, the number of tapers used.
                \codeitem {mtpsd\_workspace}{getinfo()}
                    Returns a structure that contains all computation parameters used by the \code{mtpsd} class.
            \end{compactcodelist}
        }\\
        \hline
   \end{tabular}
\end{table}

\subsubsection{The \texttt{mtpsd\_workspace} Structure}

\index{mtpsd\_workspace@\texttt{mtpsd\_workspace}|(}For more control over the multitaper parameters, a special \code{mtpsd\_workspace} structure is available.  It's definition is outlined in Listing \ref{lst:mtpsdwork}.  The first six members of the structure specify the multitaper computation parameters.  The last two are set automatically by the \code{mtpsd} class.  The structure has two constructor methods to initialize default values: one takes no inputs, and the other accepts the data length and the time-bandwidth product.
\smallskip

\begin{lstlisting}[label=lst:mtpsdwork,caption=The \texttt{mtpsd\_workspace} structure]
enum WEIGHT_METHOD {ADAPT, EIGEN, EQUAL};
enum DATA_TYPE {REAL_DATA, COMPLEX_DATA};

struct mtpsd_workspace{
    // should be set manually                                [default]       
    uint_t n;                     // length of data             [0] 
    double nW;                    // time-bandwidth product     [1]
    WEIGHT_METHOD weight_method;  // type of weighting          [ADAPT]
    uint_t N;                     // length of the DFTs         [n]
    uint_t K;                     // number of tapers           [2nW-1]
    double Fs;                    // sampling frequency         [1]
    bool remove_mean;             // specifies if weighted      [true]
                                  //   means to be removed
    // set automatically
    DATA_TYPE dtype;              // data type (real or complex)
    uint_t nwk;                   // # independent wts per Sk

    // constructors
    mtpsd_workspace(): n(0), nW(1), weight_method(ADAPT), N(0),/*@\lstomitnum\lstskipnum{98}@*/
                   K(2), Fs(1), remove_mean(true),             /*@\lstomitnum@*/
                   dtype(REAL_DATA), nwk(0){ }                 /*@\lstskipnum{-100}@*/
    mtpsd_workspace(uint_t n_, double nW_):  n(n_), nW(nW_),   /*@\lstomitnum\lstskipnum{100}@*/
                   weight_method(ADAPT), N(n_), K(floor(2*nW_)-1),/*@\lstomitnum@*/
                   Fs(1), remove_mean(true), dtype(REAL_DATA),   /*@\lstomitnum@*/
                   nwk(n_){ }                                   /*@\lstskipnum{-103}@*/
};
\end{lstlisting}
\bigskip

Listing \ref{lst:workexample} shows an example of a workspace being used to initialize an \code{mtpsd} object.  If any of the supplied parameters are invalid, the \code{mtpsd} constructor will use the default values.  Alternatively, the \code{mtpsd\_workspace} can first be corrected by supplying it to the function:
\begin{lstplainblock}
void fix_workspace( mtpsd_workspace &myworkspace ).
\end{lstplainblock}\index{fix\_workspace@\texttt{fix\_workspace()}}
This will modify the invalid parameters in the supplied workspace directly.

\pagebreak

\begin{lstlisting}[label=lst:workexample,caption=Example using the \texttt{mtpsd\_workspace} ]
#include "mtpsd.h"              /*@\lstomitnum@*/
...                             /*@\lstskipnum{8}@*/
{
    fftw_complex *x;            // complex time series /*@\lstomitnum@*/
    ...                         /*@\lstskipnum{7}\lstomitnum@*/

    mtpsd_workspace params;

    params.n=100;               // x has 100 points
    params.nW=3.5;
    params.weight_method=EIGEN; // eigenvalue weighting
    params.N=512;               // 512-point spectrum
    params.K=10;                
    params.Fs=1000;             // 1 kHz
    params.remove_mean=false;   // mean not removed

    mtpsd spectrum<fftw_complex>(x, params);
    try{
        spectrum.compute();
    }
    catch(...){}                /*@\lstomitnum@*/
    ...                         /*@\lstskipnum{8}@*/
}
\end{lstlisting}
\index{mtpsd\_workspace@\texttt{mtpsd\_workspace}|)}

\subsubsection{Supplying Custom Tapers}

It is also possible to supply your own tapers when initializing an \code{mtpsd} object.  This allows the user to have full control over the way the tapers are computed.  The tapers are assumed to be rows of a \code{double} array stored in row-major format.  They can be passed, along with their energy concentrations, to one of the following two constructors:
\smallskip

\begin{lstplainblock}
mtpsd(T *data, const double *tapers, const double *lambda, uint_t n, uint_t K)
mtpsd(T *data, const double *tapers, const double *lambda, mtpsd_workspace work)
\end{lstplainblock}
\smallskip

\noindent Ideally, the tapers should be discrete prolate spheroidal sequences, which can be calculated using the \code{dpss} library (covered in Section \ref{sec:dpss}).  This is how the \code{mtpsd} class computes tapers in the first two constructors (Listing \ref{lst:mtpsdclass}:\ref{lst:mtpsdconstruct1}--\ref{lst:mtpsdconstruct2}).  However, the algorithms will work for any tapers, as long as they are scaled to have a unit energy:
$\|h_k\|^2 = 1\; \forall\, k$.  

For non-DPSS tapers, the eigenvalue and adaptive weighting schemes are no longer guaranteed to reduce broad-band bias.  The eigenvalues can instead be interpreted as relative weights.  For example, an \code{mtpsd} object can be tricked into computing a periodogram\index{periodogram} by providing:
\begin{align*}
    h_0(i) & =\dfrac{1}{\sqrt{n}}, & \quad h_1(i)& =0, & \text{for } i=0,\ldots,n-1,\\
    \lambda_0 &=1, & \lambda_1 &=0,
\end{align*}
and using the eigenvalue weighting scheme. \index{mtpsd class@\texttt  {mtpsd} class|)}

\subsection{Advanced Use \label{sec:advanced}}

The spectrum can also be computed by manually building arrays and calling methods in the \code{mtpsd} library to fill them.  This gives the user more control over memory management, and allows for more customization when it comes to weights and tolerances.

Brief descriptions for the advanced methods are given in the following sections.  None of these methods create any substantial arrays on the heap; with the exception of a few local variables and one array of length $K$, all working memory, inputs and outputs are supplied by the user.  A list and description of the common variables, needed by several of the functions, is given in Table \ref{tbl:mtpsdcommonvars}.
\bigskip

\begin{table}[!hb]
    \centering
    \caption{Common variables in the \texttt{mtpsd} Library \label{tbl:mtpsdcommonvars}}
    \renewcommand\arraystretch{1.3}
    \vspace*{-1.5ex}
    \small
    \begin{tabular}{r@{\hspace{1.5ex}}p{5ex}@{\hspace{2ex}}p{0.65\textwidth}}
        \hline\hline
        \code{T*} & \code{x} & pointer to the time-series of which to compute the spectrum.  \code{T} is a template parameter that is either \code{double} or \code{fftw\_complex} (\code{double \_\_complex\_\_}).\\
        \code{uint\_t} & \code{n} & the number of data points (unsigned integer).\\
        \code{uint\_t} & \code{K} & the number of data tapers.\\
        \code{const double*} & \code{h} & pointer to the array of length-$n$ tapers, stored in row-major format.  Each taper is assumed to occupy one row.  \code{h} should have $K\times n$ elements.\\
        \code{const double*} & \code{l} & pointer to the array of energy concentrations (eigenvalues) of the tapers.  \code{l} should have length $K$.\\
        \code{uint\_t} & \code{N} & the number of points in the frequency domain, greater than or equal to $n$.  This specifies the size of the FFT when computing the eigenspectra.\\
        \code{double*} & \code{S} & pointer to the estimate of the power spectral density function.\\
        \code{double*} & \code{Sk} & pointer to the eigenspectra array, stored in row-major format.\\
        \code{fftw\_complex*} & \code{Jk} & pointer to the eigencoefficient array, stored in row-major format.\\
        \code{uint\_t} & \code{nwk} & the number of independent weights per eigenspectrum.  For most purposes, \code{nwk} will either be $1$ (equal or eigenvalue weighting) or $N$ (adaptive).\\
        \code{double*} & \code{wk} & pointer to the array of weights.  \code{wk} has size $K\times$\code{nwk}, stored in row-major format.  The weight \code{wk[k*nwk+i\%nwk]} is applied to \code{Sk[k*N+i]} when calculating \code{S[i]}.\\
        \code{uint\_t} &\code{nv} & the number of elements in the degrees of freedom array.  This is usually $1$ (if $\nu$ is independent of frequency) or $N$ ($\nu$ is frequency-dependent), and should be equal to \code{nwk}. \\
        \code{double*} & \code{v} & pointer to the degrees of freedom array.  \code{v} has length \code{nv}, where \code{v[i\%nv]} is the equivalent degrees of freedom of the estimate \code{S[i]}, assuming a scaled $\chi^2_\nu$ distribution.\\
        \hline
    \end{tabular}
\end{table}

\subsubsection{Computing the Eigencoefficients/Eigenspectra}

Recall from Section \ref{sec:tapers} that the eigenspectra are just the squared-magnitude of the eigencoefficients:
\begin{align}
    \hat{S}_k(f) & = \|\hat{J}_k(f)\|^2.    \tag{\ref{eq:eigencoeffs}}
\end{align}
The eigencoefficients are needed by the F-test, and are useful for removing line frequencies (see, for example, \cite{percival:multitaper}).  Since the eigenspectra are readily obtained using Equation \ref{eq:eigencoeffs}, it is not necessary to store both arrays.  To potentially save memory, the code in this library has been written so that $\hat{S}_k$ is not explicitly required.

Since the input data can be either real or complex, the methods for computing eigencoefficients and eigenspectra are written using templates, taking possible values \code{T=double} or \code{T=fftw_complex}:
\medskip

\begin{lstshortblock}
template <class T>
void eigencoeffs(T *x, uint_t n, const double *h, const double *l, 
                 uint_t K, bool remove_mean, uint_t N, fftw_complex *Jk)
\end{lstshortblock}\index{eigencoeffs@\texttt{eigencoeffs()}}
\begin{fdescription}
 Inputs: & \code{x, n, h, l, K, remove_mean, N}\\
 Outputs: & \code{Jk}\\
  & Computes the eigencoefficients using FFTs of length $N$.  The boolean variable \code{remove_mean} specifies whether or not to remove the weighted means\index{weighted mean} from the data before tapering (see section \ref{sec:removemean}).
\end{fdescription}
\medskip

\index{eigenspectra@\texttt{eigenspectra()}}
\begin{lstshortblock}
template <class T>
void eigenspectra(T *x, uint_t n, const double *h, const double *l, 
                 uint_t K, bool remove_mean, uint_t N, fftw_complex *Jk, 
                 double *Sk)
\end{lstshortblock}
\begin{fdescription}
    Inputs: & \code{x, n, h, l, K, remove_mean, N}\\
    Outputs: & \code{Jk, Sk}\\
    & Computes the eigencoefficients \emph{and} eigenspectra using FFTs of length $N$.  The boolean variable \code{remove_mean} specifies whether or not to remove the weighted means from the data before tapering.
\end{fdescription}
\fdbottom

\subsubsection{Computing the Spectrum}

In order to estimate the overall spectrum, a set of weights is required.  These weights \emph{must} be normalized so that
\begin{align*}
    \sum_{k=0}^{K-1}w_k(i)=1, \quad \forall\,i.
\end{align*}
\noindent The array of weights, \code{wk}, is assumed to be a row-major representation of a two-dimensional array.  Each row consists of \code{nwk} elements, where \code{wk[k*nwk+i\%nwk]} is the weight applied to the $k$th eigenspectrum at the $i$th frequency.  Linear weighting schemes have \code{nwk=1}, since the weights are frequency independent.  Adaptive weighting has \code{nwk=N}.  

The following methods compute the spectrum given a set of weights and either the eigencoefficients or the eigenspectra.
\medskip

\begin{lstshortblock}
void combine_eig_coeffs( const fftw_complex *Jk, uint_t K, uint_t N, 
                         const double *wk, uint_t nwk, double *S)
void combine_eig_spec  ( const double *Sk, uint_t K, uint_t N, 
                         const double *wk, uint_t nwk, double *S)
\end{lstshortblock}\index{combine\_eig\_coeffs@\texttt{combine\_eig\_coeffs()}}\index{combine\_eig\_spec@\texttt{combine\_eig\_spec()}}
\begin{fdescription}
    Inputs: & \code{Jk} or \code{Sk, K, N, wk, nwk}\\
    Outputs: & \code{S}\\
    & Computes the multitaper spectrum estimate using the supplied weights and either the eigencoefficients or the eigenspectra.
\end{fdescription}
\medskip

\begin{lstshortblock}
void combine_eig_coeffs( const fftw_complex *Jk, uint_t K, uint_t N, 
                         const double *wk, uint_t nwk, double *S, 
                         double &diff)
void combine_eig_spec  ( const double *Sk, uint_t K, uint_t N, 
                         const double *wk, uint_t nwk, double *S, 
                         double &diff)
\end{lstshortblock}\index{combine\_eig\_spec@\texttt{combine\_eig\_spec()}}\index{combine\_eig\_coeffs@\texttt{combine\_eig\_coeffs()}}
\begin{fdescription}
    Inputs: & \code{Jk} or \code{Sk, K, N, wk, nwk, S}\\
    Outputs: & \code{S, diff}\\
    & Computes the multitaper spectrum estimate using the supplied weights, and returns the average absolute difference from the previous contents of \code{S}: \code{diff} = $\sum|\hat{S}_\text{new}-\hat{S}_\text{old}|/N$. Note that \code{S} is both an input \emph{and} an output.  This method is useful for adaptive weighting.
\end{fdescription}
\fdbottom

\noindent For adaptive weighting\index{multitaper spectral estimator!adaptive weighting}, the weights and spectrum estimate are computed by a single routine.  The adaptive process will continue until the average change in the spectrum estimate is less than a supplied tolerance.
\medskip

\begin{lstshortblock}
void adapt_wk( double varx, const double *l, const double *Sk, uint_t K, 
               uint_t N, double tol, double *wk, double *S_init, 
               double *S)
void adapt_wk( double varx, const double *l, const fftw_complex *Jk, 
               uint_t K, uint_t N, double tol, double *wk, 
               double *S_init, double *S)
void adapt_wk( double varx, const double *l, const fftw_complex *Jk, 
               uint_t K, uint_t N, double tol, double *wk, 
               double *S)
void adapt_wk( double varx, const double *l, const double *Sk, uint_t K, 
               uint_t N, double tol, double *wk, double *S)
\end{lstshortblock}\index{adapt\_wk@\texttt{adapt\_wk()}}
\begin{fdescription}
    Inputs: & \code{varx, l, Jk} or \code{Sk, K, N, tol, (S_init)}\\
    Outputs: & \code{wk}, \code{S}\\
    & Adaptively computes the weights and the spectrum using Equations \eqref{eq:adaptweights} and \eqref{eq:adaptspec}.  \code{varx} is the estimated total power in \code{x}.  If the data is centred (or weighted means removed), then \code{varx} should be the variance of \code{x}.  Otherwise, \code{varx} should be the second non-central moment. Iterations will continue until the average absolute difference between updates (\code{diff} from \code{combine_eig_coeffs/spec}) is less than the supplied tolerance, \code{tol}.  If \code{S_init} is supplied, it is used as the initial estimate of the spectrum.  Otherwise, the initial estimate uses equal weighting of the first two eigenspectra.
\end{fdescription}
\fdbottom

\subsubsection{Confidence Intervals}\index{confidence interval|(}

For the confidence intervals, it is assumed $n$ is large enough that the individual eigenspectra follow scaled $\chi^2$ distributions.  The equivalent degrees of freedom for the combined estimate can be calculated from the vector of weights with the following:
\medskip

\begin{lstshortblock}
void degrees_of_freedom( double *wk, uint_t nwk, uint_t K, double *v)
\end{lstshortblock}\index{degrees of freedom@\texttt{degrees\_of\_freedom()}}
\begin{fdescription}
    Inputs: & \code{wk, nwk, K}\\
    Outputs: & \code{v}\\
    & Computes the equivalent degrees of freedom of the combined spectrum estimate, \code{S}.  \code{v} has length \code{nwk}, which is 1 for linear weighting schemes, and $N$ for adaptive.
\end{fdescription}
\fdbottom

\noindent The degrees of freedom\index{degrees of freedom} are sometimes useful when analyzing the variance of the estimate.  The smaller the degrees of freedom, the larger the variance, making the estimate less stable.  However, since $\nu$ can easily be computed from the weights using Equation \eqref{eq:dof}, the methods have been written so that the array \code{v} need not be stored.  The following routines can be used to compute confidence intervals.
\medskip

\begin{lstshortblock}
void confidence_factor( double p, double *v, uint_t nv , double *Cf)
void confidence_factor( double p, double *wk, uint_t nwk, uint_t K, 
                        double *Cf)
\end{lstshortblock}\index{confidence\_factor@\texttt{confidence\_factor()}}
\begin{fdescription}
    Inputs: & \code{p, v} or \code{wk}, \code{nv} or \code{nwk, K}\\
    Outputs: & \code{Cf}\\
    & Computes the lower and upper confidence factors $\nu/Q_\nu$, corresponding to a \code{p}$\times100\%$ confidence interval.  \code{Cf} is an array of length $2\times$\code{nv}.  The first row contains the lower confidence factor, and the second contains the upper.  The confidence interval can then be generated from these factors with:\\
    & \multicolumn{1}{c}{$\mathcal{C} = [\; $\code{ Cf[i\%nv]*S[i], Cf[i\%nv+nv]*S[i]} $ ]$.}
\end{fdescription}
\fdbottom
\medskip

\begin{lstshortblock}
void confidence_interval( double p, double *S, uint_t N, double *v, 
                          uint_t nv, double *Sc)
void confidence_interval( double p, double *S, uint_t N, double *wk, 
                          uint_t nwk, uint_t K, double *Sc)
\end{lstshortblock}\index{confidence\_interval@\texttt{confidence\_interval()}}
\begin{fdescription}
    Inputs: & \code{p, S, N, v} or \code{wk}, \code{nv} or \code{nwk, K}\\
    Outputs: & \code{Sc}\\
    & Computes the lower and upper confidence bounds corresponding to a \code{p}$\times100\%$ interval.  \code{Sf} is an array of length $2\times$\code{N}.  The first row contains the lower confidence bound, and the second contains the upper.
\end{fdescription}
\fdbottom

\noindent For linear weighting schemes, storing the confidence factors instead of the intervals can save memory: the factor is frequency-independent, so only two values need to be computed and stored.
\index{confidence interval|)}

\subsubsection{The F-test}

\index{F-test|(}The implemented F-statistic is weight-dependent, which differs from Percival and Walden's version \cite{percival:multitaper}.  The same underlying assumptions are imposed, but the degrees of freedom and the estimated Fourier coefficients now incorporate non-equal weights.
\medskip

\begin{lstshortblock}
void F_statistic( fftw_complex *Jk, uint_t N, uint_t K, double *wk, 
                  uint_t nwk, double *h, uint_t n, double *F)
\end{lstshortblock}\index{F\_statistic@\texttt{F\_statistic()}}
\begin{fdescription}
    Inputs: & \code{Jk, N, K, wk, nwk, K, h, n}\\
    Outputs: & \code{F}\\
    & Computes the F-statistic described in Section \ref{sec:ftest}, Equation \eqref{eq:Fdist}. \code{F} has length \code{N}.
\end{fdescription}
\medskip

\begin{lstshortblock}
void F_threshold( double p, double *wk, uint_t nwk, uint_t K, double *Fu)
void F_threshold( double p, double *v, uint_t nv, double *Fu)
\end{lstshortblock}\index{F\_threshold@\texttt{F\_threshold()}}
\begin{fdescription}
    Inputs: & \code{p, v} or \code{wk, nv} or \code{nwk, K}  \\
    Outputs: & \code{Fu}\\
    & Computes the \code{p}$\times100\%$ upper threshold for the F-test.  \code{Fu} has length \code{nv}.  This threshold may take the value \code{INFINITY} if the number of degrees of freedom is too small.  This can occur in adaptive weighting at a frequency deemed to be heavily biased.
\end{fdescription}
\fdbottom

\noindent The actual F-test can be performed by checking if \code{F[i] > Fu[i\%nv]}.  When this inequality is satisfied, the $i$th frequency is deemed significant.
\index{F-test|)}

\subsubsection{An Advanced Example}

The following is an example that uses the individual methods from this section to compute an adaptive spectrum estimate, confidence interval, and F-test.
\smallskip

\begin{lstlisting}[label=lst:mtpsdadvanced,caption=An \texttt{mtpsd} advanced example ]
#include "mtpsd.h"              /*@\lstomitnum@*/
...                             /*@\lstskipnum{8}@*/
{
    // allocate memory
    double *x = new double[n];
    double *h = new double[n*K];
    double *l = new double[K]; 
    double *S = new double[N];
    fftw_complex *Jk = (fftw_complex *)fftw_malloc(sizeof(fftw_complex)*N*K);
    double *wk = new double[N*K];   // adaptive, so nwk=N
    double *Sc = new double[2*N];
    double *F = new double[N];
    double *Fu = new double[N];
    
    // Fill x, h, and l                                         /*@\lstomitnum\lstskipnum{7}@*/
    ...                                                         /*@\lstomitnum@*/

    // calculate eigencoefficients, removing the mean
    eigencoeffs<double>(x, n, h, l, K, true, N, Jk);
    
    // build initial estimate
    wk[0]=l[0]/(l[0]+l[1]);
    wk[1]=l[1]/(l[0]+l[1]);
    combine_eig_coeffs(Jk, 2, N, wk, 1, S);     //eigenvalue weights for S0,S1

    // adaptive weighting
    double varx = var<double>(x, n);            // mom2() if remove_mean=false
    double tol = 1e-8;
    adapt_wk( varx, l, Jk, K, N, tol, wk, S, S);  

    // 95% confidence interval
    confidence_interval(0.95, S, N, wk, N, K, Sc);

    // 99% F-test
    F_statistic(Jk, N, K, wk, N, h, n, F);          
    F_threshold(0.99, wk, N, K, Fu);     

    for (uint_t ii=0; ii<N; ii++){
        if ( F[ii] > Fu[ii] )
            printf( "Frequency %d is significant!\n", ii);
    }                               /*@\lstomitnum\lstskipnum{7} @*/
    ...                             /*@\lstomitnum@*/

    // cleanup
    delete [] x;  delete [] h;  delete [] l; delete [] S;  delete [] wk; 
    delete [] Sc; delete [] F;  delete [] Fu;
    fftw_free(Jk);    // fftw-safe memory freeing
}
\end{lstlisting} \index{mom2@\texttt  {mom2()}}\index{var@\texttt  {var()}}\index{eigencoeffs@\texttt{eigencoeffs()}}\index{combine\_eig\_coeffs@\texttt{combine\_eig\_coeffs()}}\index{confidence\_interval@\texttt{confidence\_interval()}}\index{adapt\_wk@\texttt{adapt\_wk()}}\index{F\_statistic@\texttt{F\_statistic()}}\index{F\_threshold@\texttt{F\_threshold()}}
\index{mtpsd@\texttt{mtpsd}|)}